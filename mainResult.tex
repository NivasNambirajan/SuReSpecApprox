\section{Main Result}
Let $C \in \reals^{m \times n}, \; m \geq n$ and let $B \in \reals^{m \times r}$ be a basis for $\B$, an $r$-dimensional subspace of $\reals^m$. Any matrix of rank $k \leq r$ with columns in $\B$ may be written is $Y = BZ$, with $rank(Z) = k$. We define the critical rank, $k^*$, as the number of singular values of $C$ strictly larger than the orthogonal complement of $C$ with respect to $\B$ in (spectral) norm:
\begin{eqnarray}\label{critical rank inequality}
\sigma_{k^*}(C) > \norm{C_{\N}} \geq \sigma_{k^*+1}(C).
\end{eqnarray}
Our main observation is that the best rank-$k$ approximation, $C_{\B, k}$, with columns in $\B$, to $C$ is for the most part either
\begin{itemize}
\item as good as the best rank-$k$ approximant, $C_k$, without any subspace-restrictions on its columns, or
\item as good as the best subspace restricted approximant, $C_{\B}$, without any rank restrictions.
\end{itemize}
Formally:
\begin{theorem}\label{main theorem}
Given $1 \leq k < r,$
\[ \norm{C - C_{\B,k}}  \left\{ \begin{array}{ll} = \norm{C - C_k} = \sigma_{k+1}(C), & \quad k < k^*,\\
[5 pt] \leq (1+\epsilon)\norm{C_{\N}}, & \quad k^* \leq k < k^*+h, \text{ } \forall \epsilon > 0 \\
[5 pt] = \norm{C - C_{\B}} = \norm{C_{\N}}, & \quad k \geq k^*+h,  \end{array} \right. \]
where $h=\rank(C_{\B} V_0V_0^T)$ 
and \math{V_0} is a basis for the top singular subspace of $C_{\N}$.
\end{theorem}

\noindent We note that the effect of the subspace for column-restriction, $\B$, on the spectral approximation error is to place an upper bound, $k^*$, on the rank up to which we may expect to do as well as a subspace-unrestricted rank-$k$ approximation to $C$. When the rank of the approximation meets or exceeds this allowance, $k^*$, we obtain an approximation that provides the same (or almost the same) error as $C_{\B}$, the best $\B$-restricted approximation to $C$ possible. As a typical case, consider a full rank $C$. If the \emph{rank allowance}, $k^*$, provided by $\B$ is very small - $k^* \ll r$, say - then $C_{\B,  k^*}$ is as good an approximation to $C$ as $C_{\B}$ (which is of rank $r$), but with far lesser rank. If, on the other hand, $k^* = r$, we see that restricting the columns to belong to $\B$ doesn't affect the approximation at all, and $C_{\B, k \leq k^*}$ is as good as $C_k$. We refer the reader to Section 3 for the proof of Theorem~\ref{main theorem}. 

\subsection{Construction Of Subspace Restricted Low Rank Approximations}
We now turn to the construction of approximations having the properties promised in Theorem~ \ref{main theorem}. The construction-strategy, as in Sou and Rantzer, will be to remove the explicit subspace-restriction, but retain the rank-restriction, by constructing a series of transformations of, eventually leading to a problem of approximating a modified matrix appropriately. We then extract a subspace restricted low-rank approximation to MAIN by inverse-transforming the solution to this equivalent problem.

\subsubsection{Removing The Explicit Subspace Restriction}
Suppose spectral error $\beta \geq \norm{C_{\N}}$ is achievable. So
 $\exists Y = BZ$, such that $\norm{C - Y} \leq \beta$. Then,
\begin{align*}
&\phantom{\iff}\norm{C - Y} \leq \beta \\[3pt]
& \iff \left( {C} - {Y} \right)^T \left( {C} - {Y} \right) 
\Leq \beta^2 I \\
& \iff \left( {C}_N + {C}_B - {Y} \right)^T \left( {C}_N + {C}_B - {Y} \right) \Leq \beta^2 I \\
& \iff \left( {C}_B - {Y} \right)^T \left( {C}_B - {Y} \right) \Leq \beta^2 I - {C}_N ^T {C}_N, \qquad (\text{since }  {C}_N^T({C}_B - {Y})=0).
\end{align*}
Define 
$\Delta = \beta^2 I - {C}_N ^T {C}_N$, and let the \emph{full} SVD of $C_N$ be 
\[ C_N = U_N\Sigma_NV_N^T, \quad U_NU_N^T=U_N^TU_N=I_m, \quad V_NV_N^T = V_N^TV_N = I_n. \]
Then, $\Delta = \beta^2 I - {C}_N ^T {C}_N = V_N(\beta^2I - \Sigma_N^2)V_N^T,$ and can be written as
\[ \Delta =  [  V_+ \quad V_0  ]
\begin{bmatrix}
&\Sigma_+ & 0 &\\
&0 & 0&
\end{bmatrix}
\begin{bmatrix}
&V_+^{\text{T}}&\\
&V_0^{\text{T}}&
\end{bmatrix} = V_+ \Sigma_+ V_+^T \]
where the orthonormal bases $V_+$ and $V_0$ correspond to the positive and zero eigenvalues of $\Delta$ respectively. We observe that
\math{V_0} spans the subspace corresponding to the top singular
value of \math{C_N}.
We can now apply Lemma \ref{basic lemma} with the null space of 
$\Delta$ spanned by $V_0$. Thus, summarizing the results we have
\begin{lemma}\label{combined}
\begin{eqnarray}\label{equation not invert}
\left( {C}_B - {Y} \right)^T \left( {C}_B - {Y} \right) \Leq \Delta 
\text{\quad if and only if \quad }
\begin{cases}
\quad (C_B - Y)V_0 = 0, \\
\norm{\left( {C}_B - {Y} \right) \sqrt{\Delta^{\dagger}} } \leq 1,
\end{cases}
\end{eqnarray}
\end{lemma}
\noindent which is a generalization of the case when \math{\Delta} is invertible (in which case \math{V_0} is set to \textbf{0} and \math{\Delta^\dagger=\Delta^{-1}}).
Using the row basis  $[V_+ \quad V_0]^T$, we can write
%\begin{align*}
\[Y = Z_1V_+^T + Z_2V_0^T\]
%\end{align*}
Then, from (\ref{equation not invert})
\begin{align*}
(C_B - Y)V_0 = 0 & \iff (C_B - Z_1V_+^T - Z_2V_0^T)V_0 = 0\\
& \iff C_BV_0 - Z_2 = 0, \quad V_+ \perp V_0\\
& \iff Z_2 = C_BV_0\\
& \iff Y = Z_1V_+^T + C_BV_0V_0^T
\end{align*}
Using this $Y$ and the second condition in (\ref{equation not invert})
\begin{align*}
\quad \norm{\left( {C}_B - {Y} \right) V_+ {\Sigma}^{-1/2}_+V_+^T } \leq 1 \quad \iff & \quad \norm{\left( {C}_B - C_BV_0V_0^T - Z_1V_+^T \right) V_+ {\Sigma}^{-1/2}_+V_+^T } \leq 1,\\
\iff & \quad \norm{\left(C_BV_+V_+^T - Z_1V_+^T \right) V_+ {\Sigma}^{-1/2}_+V_+^T } \leq 1,\\
\iff & \quad \norm{C_BV_+{\Sigma}^{-1/2}_+V_+^T  - Z_1{\Sigma}^{-1/2}_+V_+^T } \leq 1\\
\iff & \quad \norm{\tilde{C} - \tilde{Y}} \leq 1
\end{align*}
where, \quad $\tilde{C} = C_BV_+{\Sigma}^{-1/2}_+V_+^T = C_B \sqrt{\Delta^{\dagger}}$, \quad and \quad $\tilde{Y} = Z_1{\Sigma}^{-1/2}_+V_+^T$.
Since the row, column spaces of $\tilde Y$ are contained in the row, column spaces of $\tilde C$, we may obtain $\tilde{Y}$ that satisfies the above condition by gathering all the singular components of $\tilde{C}$ with singular values strictly larger than 1. 

\subsubsection{Extracting The Subspace-Restricted Low Rank Approximation}
Since \math{V_+,V_0} are mutually orthogonal,
$$\rank(Y)=\rank(Z_1V_+^T)+\rank(C_BV_0V_0^T)=\rank(Z_1V_+^T)+h.$$
So, if error \math{\beta} can be achieved with rank 
\math{k+h}, we pick $\tilde Y$ as $(\tilde C)_k = (C_B\sqrt{\Delta^{\dagger}} )_k$.  We note that 
\[ Z_1 V_+ ^T = Z_1 {\Sigma}^{-1/2}_+V_+ ^T V_+ \Sigma^{1/2} V_+ ^T = \tilde Y \sqrt{\Delta},  \]
and $Y$ can be constructed as
\[ Y = \tilde Y \sqrt{\Delta} + C_B V_0 V_0 ^T. \]
%We show that this truncation ensures $\rank(\tilde{Y}) \leq k^*$ (Lemma \ref{rank Y tilde}). Therefore, gathering a $k^*$ number of singular components of $\tilde{C}$ to form $\tilde{Y}$ is sufficient to guarantee a solution $Y$ that achieves minimum error $\norm{C_N}$. 
We summarize our discussion in the following lemma.
\begin{lemma}\label{achievable}
Let \math{\beta\ge\norm{C_N}} and let \math{\Delta=\beta^2I-C_N^TC_N} with
\math{V_0} as the basis for its nullspace. Let 
\math{h=\rank(C_BV_0V_0^T)} (if \math{\Delta} is invertible then
\math{V_0=\bm0} and \math{h=0}), and suppose error \math{\beta} can be achieved with 
rank \math{k+h}.
Then 
$$
Y=(C_B \sqrt{\Delta^{\dagger}})_k \sqrt{\Delta} +C_BV_0V_0^T
$$
is a solution satisfying \math{\norm{C-Y}\le\beta}, 
\math{\rank(Y)\le k+h} and \math{Y=BX} for some \math{X\in\RR^{r\times n}}.
\end{lemma}

\subsubsection{The Algorithm}
An algorithm that solves problem (\ref{main problem}) results naturally from our analysis above. This algorithm is presented as \textbf{Su}bspace \textbf{Re}stricted \textbf{S}pectral \textbf{A}pproximation below. It is split into modules both for the purposes of exposition and to serve as units of improvement in runtime of the overall algorithm. We also provide the leading running times - running times of operations that are $\Omega(T_{svd(B)})$ - next to the corresponding operations.
%% ============ The Algorithm begins ==================
\begin{framed}
\begin{alg}\label{alg:outline}
{\bf SuReSpectralApprox($C$, $B$, $k$, $\epsilon$) }
\end{alg}
\noindent {\bf Input:} $C \in \reals^{m \times n}$, $B \in \reals^{m \times r}$, $k$ : $1 \leq k \leq n$, and $\epsilon > 0$.\\
\noindent {\bf Output:} $Y$ such that $Y=BX$, $\rank(Y) \leq k$, with error $\beta = \norm{C - Y}$ achieved according to Theorem \ref{main theorem}.
%\noindent {\bf Steps:}
%
\begin{align*}
1. \quad& [C_{\B}, C_{\N}] = \textbf{Orthosplit}(C, B) & O(mnr + T_{svd(B)})\\
2. \quad& h = \textbf{rank}(C_{\B} V_0 V_0 ^T)  & O\left(T_{svd(B)} \right)\\
3. \quad& [\beta,\bar h] = \textbf{SetError}(C, C_{\N}, k, h, \epsilon) & O\left( T_{svd(C)} \right) \\
4. \quad& \Delta = \beta^2 I - C_{\N} ^T C_{\N} & \\
5. \quad& Y = \textbf{ExtractApproximation}(C_{\B}, \Delta, k, \bar h) & O\left( T_{svd(C)} \right) \\
\end{align*}
%
\end{framed}
%% ============ The Algorithm ends ==================

\noindent We note that the approximation error may be obtained prior to obtaining the approximation itself. The integer, $h$, is the rank-baggage for obtaining an approximation, $C_{\B, k}$, that is as good as the best possible approximation, $C_{\B}$, to $C$, with columns in $\B$. The integer $\bar{h}$ is the rank-baggage for any approximation, equaling $h$ when $\beta = \norm{C_{\N}}$ and 0 in every other case. The running time of this algorithm is $O\left( mnr + T_{svd(C)} \right)$.

\begin{framed}
\begin{alg}\label{alg:seterr}
{\bf SetError($C, C_{\N}, k, h$) }
\end{alg}
\noindent {\bf Input:} $C, C_{\N} \in \reals^{m \times n}$ and $k, h \in \mathbb{Z^+}$\\
\noindent {\bf Output:} $\beta, \bar h$, the approximation error and rank-baggage
%\noindent {\bf Steps:}
%
\begin{align*}
1. \quad& [U_C, \Sigma_C, V_C] = \textbf{Reducedsvd}(C), \text{ where } \sigma_i := \Sigma_C (i, i)  & O\left( T_{svd(C)} \right) \\
2. \quad& \text{Compute} \norm{C_{\N}}\\
3. \quad& \text{Compute } k^* = \arg \max_i \sigma_i > \norm{C_{\N}} \\
4. \quad& \textbf{if } k < k^* \textbf{ return } [\sigma_{k+1}, 0] \\
 \quad& \textbf{if } k^* \leq k < k^* + h \textbf{ return } [(1 + \epsilon) \norm{C_{\N}}, 0] \\
\quad& \textbf{if } k \geq k^* + h \textbf{ return } [\norm{C_{\N}}, h] 
\end{align*}
%
\end{framed}


%==== Construction ====%
\begin{framed}
\begin{alg}\label{alg:extractApprox}
{\bf ExtractApproximation($C_{\B}, \Delta, k, \bar h$) }
\end{alg}
\noindent {\bf Input:} $C_{\B} \in \reals^{m \times n}, \Delta \in \reals^{n \times n}$ and $k, \bar h \in \mathbb{N}$\\
\noindent {\bf Output:} $Y$, such that $Y=BX$, $\rank(Y) \leq k$, and error $\beta = \norm{C - Y}$ achieved according to Theorem \ref{main theorem}.
%\noindent {\bf Steps:}
%
\begin{align*}
1. \quad& [V_{\Delta} \Sigma_{\Delta} V_{\Delta}] = \textbf{Reducedsvd} (\Delta),\text{ where } V_{\Delta} = [V_+, V_0],\; \Sigma_{\Delta} = \begin{pmatrix} \Sigma_+ & 0 \\ 0 & \Sigma_0 \end{pmatrix} & O\left(T_{svd(C)} \right)\\
2. \quad& R = \textbf{rankTruncate}\left( C_{\B} \; V_+ \sqrt{\Sigma_+}^{-1} V_+ ^T, k - \bar h \right)  & O\left( T_{svd(B)} \right)\\
3. \quad& Y = R \; V_+ \sqrt{\Sigma_+} V_+ ^T  + C_{\B} V_0 V_0 ^T.
\end{align*}
%
\end{framed}
%===End Construction===%
\noindent Here, $\textbf{rankTruncate}(G, a)$ is the top $a$ SVD truncation of $G$. We note that, for $k < k^* + h$, the rank-$h$ additive correction, $C_{\B} V_0 V_0 ^T$ is $\mathbf{0}$. This additive correction only features non-trivially when obtaining an approximation to $C$ that is as good as the projection, $C_{\B}$, but possibly of lower rank than $C_{\B}$. This, for instance, happens when $C$ is full rank and $k + h < r = rank(B) = rank(C_{\B})$. 