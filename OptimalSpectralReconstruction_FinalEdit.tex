\documentclass[11pt]{article}

\usepackage{mathtools,bm}
\newcommand{\RR}{\mathbb{R}}
\newcommand{\norms}[1]{{\lVert#1\rVert}^2}
\newcommand{\snorms}[1]{{\lVert#1\rVert}_2^2}
\newcommand{\rowmatrix}[4]{
\begin{pmatrix}
#1 & #2 \\
#3 & #4
\end{pmatrix}
}
\newcommand{\keywords}[1]{\emph{Keywords:} #1}
\newcommand{\trace}{\operatorname{trace}}
\newcommand{\rank}{\operatorname{rank}}
\newcommand{\card}[1]{\lvert{#1}\rvert}
\newcommand{\giventhat}{\: / \:}
\newcommand{\proj}{\pi}
\newcommand{\NN}{\mathbb{N}}

\usepackage{fullpage,amssymb,amsmath,amsthm,framed,color}
\usepackage{hyperref}

\newtheorem{theorem}{Theorem}
\newtheorem{prop}[theorem]{Proposition}
\newtheorem{lemma}[theorem]{Lemma}
\newtheorem{definition}[theorem]{Definition}
\newtheorem{conj}[theorem]{Conjecture}
\newtheorem{coro}[theorem]{Corollary}
\newtheorem{alg}{Algorithm}
\newtheorem{sylvester}[theorem]{Sylvester's Law of Inertia}
\newtheorem{transformationlemma}[theorem]{Transformation Lemma}
\newtheorem{equivalencelemma}[theorem]{Low Rank Equivalence Lemma}
\newtheorem{suffranklemma}[theorem]{Rank Sufficiency Lemma}

\renewcommand{\math}[1]{$#1$}
\providecommand{\remove}[1]{}
\def\reals{\mathbb{R}}
\def\eps{\epsilon}
\def\suchthat{\;:\;}
\def\given{\;|\;}

\newcommand{\tr}[1]{\operatorname{trace}\left(#1\right)}
\newcommand{\deter}[1]{\operatorname{det}\left(#1\right)}
\newcommand{\linspan}[1]{\operatorname{span}\left(#1\right)}
\newcommand{\vol}[1]{\operatorname{vol}\left(#1\right)}
\newcommand{\conv}[1]{\operatorname{conv}\left(#1\right)}
\newcommand{\norm}[1]{\left\|#1\right\|}
\newcommand{\frob}[1]{\left\|#1\right\|_{F}}
\newcommand{\prob}[1]{{\sf Pr} \left(#1\right)}
\newcommand{\expec}[1]{{\sf E} \left[#1\right]}
\newcommand{\abs}[1]{\left|#1\right|}
\newcommand{\size}[1]{\left|#1\right|}
\newcommand{\inner}[2]{\left\langle#1, #2\right\rangle}

\newcommand{\comment}[1]{{\sf\textcolor{blue}{#1}}}
\newcommand{\email}[1]{\href{mailto:#1}{\texttt{#1}}}

\DeclareMathOperator*{\argmin}{arg\,min}
\newcommand{\pinv}[1]{{#1}^{\dagger}}
\newcommand{\Leq}{\preceq}
\newcommand{\Geq}{\succeq}

\newcommand{\B}{\mathcal{B}}
\newcommand{\J}{\mathcal{J}}
\newcommand{\E}{\mathcal{E}}
\newcommand{\U}{\mathcal{U}}
\newcommand{\N}{\mathcal{N}}

\newcommand{\G}{\Gamma}

\newcommand{\0}{\ensuremath{\mathbf{0}}}
\renewcommand{\>}{\succ}
\newcommand{\<}{\prec}

\newcommand\numberthis{\addtocounter{equation}{1}\tag{\theequation}}

\renewcommand{\iff}{\ensuremath{\Leftrightarrow}}

%\title{Optimal Matrix Approximation  in the Spectral Norm (Draft) \\ }
\title{Rank Restricted Spectral Approximation  Within a Subspace \\ }
\author {Petros Drineas \footnote{Computer Science, Rensselaer Polytechnic Institute. Email: drinep@cs.rpi.edu} \\
\and
Abhisek Kundu \footnote{Computer Science, Rensselaer Polytechnic Institute. Email: kundua2@rpi.edu} \\
\and
Malik Magdon-Ismail \footnote{Computer Science, Rensselaer Polytechnic Institute. Email: magdon@cs.rpi.edu} \\
\and
Srinivas Nambirajan \footnote{Mathematics, Rensselaer Polytechnic Institute. Email: nambis2@rpi.edu} \\
\date{}
}
\begin{document}         
\maketitle
\noindent \textbf{\small Abstract}
\\
\\ 
Let $C \in \reals^{m \times n}$ be a real matrix and $B \in \reals^{m \times r}$ be a matrix with orthogonal columns.
We consider the classic problem of low-rank spectral approximation of $C$ with columns constrained to belong to the range of $B$. The computationally interesting case is when $r = o(n)$. While a similarly constrained approximation under the frobenius norm can be computed efficiently in $O(T_{svd(B)}+ mnr)$, there is no known algorithm for subspace constrained low-rank approximation under the spectral norm. We give an $O(T_{svd(C)})$ algorithm to find the best rank-$k$ spectral approximation of~$C$, within the range of $B$. It is open whether this problem can be solved it $O(T_{svd(B)})$.

\section{Introduction}
The best unrestricted rank-$k$ approximation to a matrix $C \in \reals^{m \times n}$ is $C_k$ in both spectral and Frobenius norms, where $C_k$ is the projection of $C$ onto its top-$k$ singular vectors (we use standard matrix notation which we summarize in section 1.1). Therefore, when $B \in \reals^{m \times r}$ spans the top $k$ singular vectors of $C$, $C_k$ is the best rank-$k$ approximation to $C$ having columns in the range of $B$. The approximation error in this case is $\sigma_{k+1}(C)$, the $(k+1)$-th largest singular value of $C$. 
However, when $B$ spans a more general subspace, $\B$, there is no known algorithm to compute the best rank-$k$ approximation of $C$ with columns in $\B$. A natural approach is to first project $C$ onto $\B$, and then take the best rank-$k$ approximation of this projection. This simple approach gives the best rank-$k$ approximation to $C$ with columns in $\B$ under the Frobenius norm (lemma 7 in \cite{BDM}), and the running time is $O(T_{svd(B)}+ mnr)$; but this is only known to give a $\sqrt{2}$-approximation to the best subspace constrained rank-$k$ approximation in spectral norm (lemma 7 in \cite{BDM}). 
Here, we show how to compute this optimal spectral approximation. Specifically, we provide a deterministic algorithm that constructs $Z = B \G$ with the property that $Z \in \reals^{m \times n}, \G \in \reals^{r \times n}$, which solves

\begin{equation}\label{main problem}
\min_{\stackrel{\G \in  \reals^{r \times n}}{rank({\G}) \leq k}} \norm{{C} - {B\G}}
\end{equation}
Our main result is the following theorem.
\begin{theorem}\label{thm:main}
There is a deterministic algorithm (Algorithm \ref{alg:outline})
 running in time
\math{O(T_{svd(C)}+mnr)} to construct \math{Z=B \G} which solves the optimization
problem in (\ref{main problem}).
\end{theorem}
\subsection{Notation}
Matrices will be denoted by upper-case letters and we follow Parlett's convention by denoting symmetric matrices by letters symmetric about their vertical axis, excepting two common practices: matrices with orthonormal rows/columns will be denoted by $U, V$, and the spectrum-diagonal of a matrix will be denoted by $\Sigma$. If $B \in \reals^{m \times n}$ is an $m \times n$ matrix, $\B, \B'$ will denote its column space (or range) and row space respectively. Scalars will be denoted in lower case. We define other notations in-situ. 

\subsection{Matrix Background} \label{matBack}
\begin{enumerate}
\item A real matrix $C$ of rank $\rho$ has $\rho$ positive singular values, $\sigma_i, \; 1 \leq i \leq \rho$. For $p > 0$, the Schatten p-norm of $C$, $\norm{C}_{(p)}$, is given by $\left( \sum_i \sigma_i ^p \right)^{1/p}$. For the special case where $p=2$, we have the Frobenius norm of $C$, denoted by $\norm{C}_F$. The spectral/operator norm of $C$, $\norm{C}$, is given by $\displaystyle \max_i \sigma_i$. As a convention, $\norm{ \cdot }$ will denote spectral norm unless specified otherwise. 

\item The Singular Value Decomposition (SVD) of $C$ is
\begin{align*}
%\begin{eqnarray*}
C &= \underbrace{[ \quad U_k \quad U_{\rho - k} \quad ]}_\text{$U_C \in \reals^{m \times \rho}$}
\underbrace{\begin{bmatrix}
&\Sigma_k & 0 &\\
&0 & \Sigma_{\rho - k}&
\end{bmatrix}}_\text{$\Sigma_C \in \reals^{\rho \times \rho}$}
\underbrace{\begin{bmatrix}
&V_k^{\text{T}}&\\
&V_{\rho - k}^{\text{T}}&
\end{bmatrix}}_\text{$V_C^{\text{T}} \in \reals^{\rho \times n}$} = U_C\Sigma_C V_C\\
C & = C_k + C_{\rho - k}, \quad \text{ where } \quad C_k  = U_k\Sigma_kV_k^{\text{T}}, \quad  C_{\rho - k}  = U_{\rho - k}\Sigma_{\rho - k}V_{\rho -k}^{\text{T}}
%\end{eqnarray*}
\end{align*} 
The matrices $U_C = [U_k, U_{\rho - k}]$ and $V_C = [V_k, V_{\rho - k}]$ contain as orthonormal columns the left and right singular vectors of $C$ respectively. The diagonal matrix, $\Sigma$, contains the singular values, $\sigma_i > 0, 1 \leq i \leq \rho$, of $C$ in non-increasing order: $\sigma_{i} \geq \sigma_{j}\; \forall i \leq j$. In this manner, the top $k$ singular values are contained in $\Sigma_k$ and the rest contained in $\Sigma_{\rho - k}$. 

\item The best rank-$k$ approximation of $C$ is $C_k = U_k\Sigma_kV_k^{\text{T}}$ that minimizes $\norm{C - Z}_{\alpha}$ over all matrices $Z \in \reals^{m \times n}$ having rank at most $k$, in both spectral and Schatten norms - and as a result, in the Frobenius norm. 

\item The Moore-Penrose pseudo-inverse of $C$ is denoted by $C^{\dagger} = V_C\Sigma_C^{-1}U_C^{\text{T}}$. For a symmetric matrix, $A$, we mean, by $\sqrt{A}$, the unique symmetric square root of $A$ given by $U_A \sqrt{\Sigma} U_A ^T$. 

\item Given $B \in \reals^{m \times r}$, we denote by $C_{\B}$, the orthogonal projection of $C$ onto $\B$: $C_{\B} = BB^{\dagger}C$; and by $C_{\N}$, the residual of $C$ with respect to this projection: $C_{\N} = C - C_{\B}$. So $C_{\N}$ is orthogonal to $B$. 
We define the matrix $C_{\B,k} \in \reals^{m \times n}$ to be the best rank-$k$ approximation of ${C}$ with columns in $\B$:
\[ C_{\B, k} = \arg \min_{\stackrel{\G \in \reals^{r \times n}}{rank(\G) = k}} \norm{C - B\G}\]
In general, $C_{\B,k} \neq (C_{\B})_k$. 

\item The semidefinite partial ordering, $\Leq$, is defined by $X \Leq Y$ ($Y$ dominates $X$), if and only if $Y - X$ is positive semidefinite. That every restriction of $Y$ must dominate the corresponding restriction of $X$ is a useful observation, formally contained in the following lemma.
\begin{lemma}\label{basic lemma}
Let $X = F^TF$ and  $Y$ be SPSD matrices. Let SVD of $Y = V_Y\Sigma_Y V_Y^T,$ $V_Y = [V_+, V_0]$ where $V_0$ is a basis for the null space of 
$Y$ - so \math{YV_0=0}. Then, 
\[X \Leq Y \quad \text{  if and only if } \quad
\begin{cases}
\quad XV_0 = \0, \\[3pt]
\norm{F \sqrt{Y^{\dagger}} } \leq 1.
\end{cases}.
\]
\end{lemma}
\noindent This lemma is a generalization of a transformation shown in \cite{SR}.
\end{enumerate}

\section{Main Result}
Let $C \in \reals^{m \times n}, \; m \geq n$ and let $B \in \reals^{m \times r}$ be a basis for $\B$, an $r$-dimensional subspace of $\reals^m$. Any matrix of rank $k \leq r$ with columns in $\B$ may be written is $Z = B \G$, with $rank(\G) = k$. 
Our main observation is that the best rank-$k$ approximation, $C_{\B, k}$, with columns in $\B$, to $C$ is for the most part either
\begin{itemize}
\item as good as the best rank-$k$ approximant, $C_k$, without any subspace-restrictions on its columns, or
\item as good as the best subspace restricted approximant, $C_{\B}$, without any rank restrictions.
\end{itemize}
Formally: let us define the critical rank, $k^*$, as the number of singular values of $C$ strictly larger than the orthogonal complement of $C$ with respect to $\B$ in spectral norm:
\begin{eqnarray}\label{critical rank inequality}
\sigma_{k^*}(C) > \norm{C_{\N}} \geq \sigma_{k^*+1}(C).
\end{eqnarray}
Then
\begin{theorem}\label{main theorem}
Given $1 \leq k < r,$
\[ \norm{C - C_{\B,k}}  \left\{ \begin{array}{ll} = \norm{C - C_k} = \sigma_{k+1}(C), & \quad k < k^*,\\
[5 pt] \leq (1+\epsilon)\norm{C_{\N}}, & \quad k^* \leq k < k^*+h, \text{ } \forall \epsilon > 0 \\
[5 pt] = \norm{C - C_{\B}} = \norm{C_{\N}}, & \quad k \geq k^*+h,  \end{array} \right. \]
where $h=\rank(C_{\B} V_0V_0^T)$ 
and \math{V_0} is a basis for the top singular subspace of $C_{\N}$.
\end{theorem}

\noindent We note that the effect of the subspace for column-restriction, $\B$, on the spectral approximation error is to place an upper bound, $k^*$, on the rank up to which we may expect to do as well as a subspace-unrestricted rank-$k$ approximation to $C$. When the rank of the approximation meets or exceeds this allowance, $k^*$, we obtain an approximation that provides the same (or almost the same) error as $C_{\B}$, the best $\B$-restricted approximation to $C$ possible. As a typical case, consider a full rank $C$. If the \emph{rank allowance}, $k^*$, provided by $\B$ is very small - $k^* \ll r$, say - then $C_{\B,  k^*}$ is as good an approximation to $C$ as $C_{\B}$ (which is of rank $r$), but with far lesser rank. If, on the other hand, $k^* = r$, we see that restricting the columns to belong to $\B$ doesn't affect the approximation at all, and $C_{\B, k \leq k^*}$ is as good as $C_k$. We refer the reader to Section 3 for the proof of Theorem~\ref{main theorem}. 

\subsection{Construction Of Subspace Restricted Low Rank Approximations}
We now turn to the construction of approximations having the properties promised in Theorem~ \ref{main theorem}. The construction-strategy, following that of Sou and Rantzer in \cite{SR}, will be to remove the explicit subspace-restriction, but retain the rank-restriction, by effectively constructing a series of transformations of problem (\ref{main problem}) to a problem of approximating a modified matrix appropriately. We then extract a solution to (\ref{main problem}) by inverse-transforming the solution to this equivalent problem.

\subsubsection{Removing The Explicit Subspace Restriction}
Suppose spectral error $\beta \geq \norm{C_{\N}}$ is achievable. So
 $\exists Z = B \G$, such that $\norm{C - Z} \leq \beta$. Then,
\begin{align*}
&\phantom{\iff}\norm{C - Z} \leq \beta \\[3pt]
& \iff \left( {C} - {Z} \right)^T \left( {C} - {Z} \right) 
\Leq \beta^2 I \\
& \iff \left( {C}_{\N} + {C}_{\B} - {Z} \right)^T \left( {C}_{\N} + {C}_{\B} - {Z} \right) \Leq \beta^2 I \\
& \iff \left( {C}_{\B} - {Z} \right)^T \left( {C}_{\B} - {Z} \right) \Leq \beta^2 I - {C}_{\N} ^T {C}_{\N}, \qquad (\text{since }  {C}_{\N}^T({C}_{\B} - {Z})=0).
\end{align*}
Define 
$\Delta = \beta^2 I - {C}_{\N} ^T {C}_{\N}$, and let the \emph{full} SVD of $C_{\N}$ be 
\[ C_{\N} = U_N\Sigma_NV_N^T, \quad U_N U_N^T=U_N^TU_N=I_m, \quad V_NV_N^T = V_N^TV_N = I_n. \]
Then, $\Delta = \beta^2 I - {C}_{\N} ^T {C}_{\N} = V_N(\beta^2I - \Sigma_N^2)V_N^T,$ and can be written as
\[ \Delta =  [  V_+ \quad V_0  ]
\begin{bmatrix}
&\Sigma_+ & 0 &\\
&0 & 0&
\end{bmatrix}
\begin{bmatrix}
&V_+^{\text{T}}&\\
&V_0^{\text{T}}&
\end{bmatrix} = V_+ \Sigma_+ V_+^T \]
where the orthonormal bases $V_+$ and $V_0$ correspond to the positive and zero eigenvalues of $\Delta$ respectively. We observe that
\math{V_0} spans the subspace corresponding to the top singular
value of \math{C_{\N}}.
We can now apply Lemma \ref{basic lemma} with the null space of 
$\Delta$ spanned by $V_0$. Thus, summarizing the results we have
\begin{lemma}\label{combined}
\begin{eqnarray}\label{equation not invert}
\left( {C}_{\B} - {Z} \right)^T \left( {C}_{\B} - {Z} \right) \Leq \Delta 
\text{\quad if and only if \quad }
\begin{cases}
\quad (C_{\B} - Z)V_0 = 0, \\
\norm{\left( {C}_{\B} - {Z} \right) \sqrt{\Delta^{\dagger}} } \leq 1,
\end{cases}
\end{eqnarray}
\end{lemma}
\noindent which is a generalization of the case when \math{\Delta} is invertible (in which case \math{V_0} is set to \textbf{0} and \math{\Delta^\dagger=\Delta^{-1}}).
Using the row basis  $[V_+ \quad V_0]^T$, we can write
%\begin{align*}
\[Z = Z_+V_+^T + Z_0V_0^T\]
%\end{align*}
Then, from (\ref{equation not invert})
\begin{align*}
(C_{\B} - Z)V_0 = 0 & \iff (C_{\B} - Z_+V_+^T - Z_0V_0^T)V_0 = 0\\
& \iff C_{\B}V_0 - Z_0 = 0, \quad V_+ \perp V_0\\
& \iff Z_0 = C_{\B}V_0\\
& \iff Z = Z_+V_+^T + C_{\B}V_0V_0^T
\end{align*}
Using this $Z$ and the second condition in (\ref{equation not invert})
\begin{align*}
\quad \norm{\left( {C}_{\B} - {Z} \right) V_+ {\Sigma}^{-1/2}_+V_+^T } \leq 1 \quad \iff & \quad \norm{\left( {C}_{\B} - C_{\B}V_0V_0^T - Z_+V_+^T \right) V_+ {\Sigma}^{-1/2}_+V_+^T } \leq 1,\\
\iff & \quad \norm{\left(C_{\B}V_+V_+^T - Z_+V_+^T \right) V_+ {\Sigma}^{-1/2}_+V_+^T } \leq 1,\\
\iff & \quad \norm{C_{\B}V_+{\Sigma}^{-1/2}_+V_+^T  - Z_+{\Sigma}^{-1/2}_+V_+^T } \leq 1\\
\iff & \quad \norm{\tilde{C} - \tilde{Z}} \leq 1
\end{align*}
where, \quad $\tilde{C} = C_{\B}V_+{\Sigma}^{-1/2}_+V_+^T = C_{\B} \sqrt{\Delta^{\dagger}}$, \quad and \quad $\tilde{Z} = Z_+{\Sigma}^{-1/2}_+V_+^T$.
Since the row, column spaces of $\tilde Z$ are contained in the row, column spaces of $\tilde C$, we may obtain $\tilde{Z}$ that satisfies the above condition by gathering all the singular components of $\tilde{C}$ with singular values strictly larger than 1. 

\subsubsection{Extracting The Subspace-Restricted Low Rank Approximation}
Since \math{V_+,V_0} are mutually orthogonal,
$$\rank(Z)=\rank(Z_+V_+^T)+\rank(C_{\B}V_0V_0^T)=\rank(Z_+V_+^T)+h.$$
So, if error \math{\beta} can be achieved with rank 
\math{k+h}, we pick $\tilde Z$ as $(\tilde C)_k = (C_{\B}\sqrt{\Delta^{\dagger}} )_k$.  We note that 
\[ Z_+ V_+ ^T = Z_+ {\Sigma}^{-1/2}_+V_+ ^T V_+ \Sigma^{1/2} V_+ ^T = \tilde Z \sqrt{\Delta},  \]
and $Z$ can be constructed as
\[ Z = \tilde Z \sqrt{\Delta} + C_{\B} V_0 V_0 ^T. \]
%We show that this truncation ensures $\rank(\tilde{Z}) \leq k^*$ (Lemma \ref{rank Z tilde}). Therefore, gathering a $k^*$ number of singular components of $\tilde{C}$ to form $\tilde{Z}$ is sufficient to guarantee a solution $Z$ that achieves minimum error $\norm{C_{\N}}$. 
We summarize our discussion in the following lemma.
\begin{lemma}\label{achievable}
Let \math{\beta\ge\norm{C_{\N}}} and let \math{\Delta=\beta^2I-C_{\N}^TC_{\N}} with
\math{V_0} as the basis for its nullspace. Let 
\math{h=\rank(C_{\B}V_0V_0^T)} (if \math{\Delta} is invertible then
\math{V_0=\bm0} and \math{h=0}), and suppose error \math{\beta} can be achieved with 
rank \math{k+h}.
Then 
$$
Z=(C_{\B} \sqrt{\Delta^{\dagger}})_k \sqrt{\Delta} +C_{\B}V_0V_0^T
$$
is a solution satisfying \math{\norm{C-Z}\le\beta}, 
\math{\rank(Z)\le k+h} and \math{Z=B \G} for some \math{\G\in\RR^{r\times n}}.
\end{lemma}

\subsubsection{The Algorithm}
An algorithm that solves problem (\ref{main problem}) results naturally from our analysis above. This algorithm is presented as \textbf{Su}bspace \textbf{Re}stricted \textbf{Spectral} \textbf{Approx}imation below. It is split into modules both for the purposes of exposition and to serve as units of improvement in runtime of the overall algorithm. We also provide the leading running times - running times of operations that are $\Omega(T_{svd(B)})$ - next to the corresponding operations.
%% ============ The Algorithm begins ==================
\begin{framed}
\begin{alg}\label{alg:outline}
{\bf SuReSpectralApprox($C$, $B$, $k$, $\epsilon$) }
\end{alg}
\noindent {\bf Input:} $C \in \reals^{m \times n}$, $B \in \reals^{m \times r}$, $k$ : $1 \leq k \leq n$, and $\epsilon > 0$.\\
\noindent {\bf Output:} $Z$ such that $Z=B \G$, $\rank(Z) \leq k$, with error $\beta = \norm{C - Z}$ achieved according to Theorem \ref{main theorem}.
%\noindent {\bf Steps:}
%
\begin{align*}
1. \quad& [C_{\B}, C_{\N}] = \textbf{Orthosplit}(C, B) & O(mnr + T_{svd(B)})\\
2. \quad& [U_N, \Sigma _N, V_N] = \textbf{reducedSVD}(C_{\N}) & O(T_{svd(C)}) \\ 
& \text{where } \nu_i := \Sigma _N (i, i), V_0 \text{ is made of the columns of  } V_N \text{ corresponding to } \nu_1 \\
3. \quad& h = \textbf{rank}(C_{\B} V_0 V_0 ^T)  & O\left(T_{svd(B)} \right)\\
4. \quad& [\beta,\bar h] = \textbf{SetError}(C, C_{\N}, k, h, \epsilon) & O\left( T_{svd(C)} \right) \\
5. \quad& \Delta = \beta^2 I - C_{\N} ^T C_{\N} & \\
6. \quad& Z = \textbf{ExtractApproximation}(C_{\B}, \Delta, k, \bar h) & O\left( T_{svd(C)} \right) \\
\end{align*}
%
\end{framed}
%% ============ The Algorithm ends ==================

\noindent Here, $\textbf{Orthosplit}(C, B)$ obtains the orthogonal components of $C$ with respect to $\B$ as in section \ref{matBack}, 5.
We note that the approximation error may be obtained prior to obtaining the approximation itself. The integer, $h$, is the rank-baggage for obtaining an approximation, $C_{\B, k}$, that is as good as the best possible approximation, $C_{\B}$, to $C$, with columns in $\B$. The integer $\bar{h}$ is the rank-baggage for any approximation, equaling $h$ when $\beta = \norm{C_{\N}}$ and 0 in every other case. The running time of this algorithm is $O\left( mnr + T_{svd(C)} \right)$.

\begin{framed}
\begin{alg}\label{alg:seterr}
{\bf SetError($C, C_{\N}, k, h$) }
\end{alg}
\noindent {\bf Input:} $C, C_{\N} \in \reals^{m \times n}$ and $k, h \in \mathbb{Z^+}$\\
\noindent {\bf Output:} $\beta, \bar h$, the approximation error and rank-baggage
%\noindent {\bf Steps:}
%
\begin{align*}
1. \quad& [U_C, \Sigma_C, V_C] = \textbf{reducedSVD}(C), \text{ where } \sigma_i := \Sigma_C (i, i)  & O\left( T_{svd(C)} \right) \\
2. \quad& \text{Compute} \norm{C_{\N}}\\
3. \quad& \text{Compute } k^* = \arg \max_i \sigma_i > \norm{C_{\N}} \\
4. \quad& \textbf{if } k < k^* \textbf{ return } [\sigma_{k+1}, 0] \\
 \quad& \textbf{if } k^* \leq k < k^* + h \textbf{ return } [(1 + \epsilon) \norm{C_{\N}}, 0] \\
\quad& \textbf{if } k \geq k^* + h \textbf{ return } [\norm{C_{\N}}, h] 
\end{align*}
%
\end{framed}


%==== Construction ====%
\begin{framed}
\begin{alg}\label{alg:extractApprox}
{\bf ExtractApproximation($C_{\B}, \Delta, k, \bar h$) }
\end{alg}
\noindent {\bf Input:} $C_{\B} \in \reals^{m \times n}, \Delta \in \reals^{n \times n}$ and $k, \bar h \in \mathbb{N}$\\
\noindent {\bf Output:} $Z$, such that $Z=B \G$, $\rank(Z) \leq k$, and error $\beta = \norm{C - Z}$ achieved according to Theorem \ref{main theorem}.
%\noindent {\bf Steps:}
%
\begin{align*}
1. \quad& [V_+, \Sigma_+, V_+] = \textbf{reducedSVD} (\Delta),\text{ where } \Sigma_+ \text{ has a positive diagonal } & O\left(T_{svd(C)} \right)\\
2. \quad& R = \textbf{SVtruncate}\left( C_{\B} \; V_+ \sqrt{\Sigma_+}^{-1} V_+ ^T, 1 \right)  & O\left( T_{svd(B)} \right)\\
3. \quad& Z = R \; V_+ \sqrt{\Sigma_+} V_+ ^T  + C_{\B} V_0 V_0 ^T.
\end{align*}
%
\end{framed}
%===End Construction===%
\noindent Here, $\textbf{SVruncate}(G, \alpha)$ is the top $\tau$ SVD truncation of $G$ where $\sigma_{\tau} (G) \geq \alpha$ and $\sigma_{\tau + 1} (G) < \alpha$.
We note that, for $k < k^* + h$, the rank-$h$ additive correction, $C_{\B} V_0 V_0 ^T$ is $\mathbf{0}$. This additive correction only features non-trivially when obtaining an approximation to $C$ that is as good as the projection, $C_{\B}$, but possibly of lower rank than $C_{\B}$. This, for instance, happens when $C$ is full rank and $k + h < r = rank(B) = rank(C_{\B})$. 
\section{Proofs}
\subsection{Lemma \ref{basic lemma}}
The proof of lemma \ref{basic lemma} is intuitive; we state the intuition first, followed by a formal proof. We first note that $X \Leq Y$ implies that the action of $Y$ is at least as large as the action of $X$ everywhere in $\RR^n$. Clearly, this implies that, $X$ cannot have a non-zero action in the space where $Y$ has zero action, leading to $X V_0 = \0$. This also implies that, in the space where $Y$ has non-zero action in general, its pseudo inverse must reduce a vector more than $X$ magnifies it, which is essentially the intuition contained in $\norm{F \sqrt{Y^{\dagger}} } \leq 1$. 

\begin{proof}
We prove the equivalence between $X \Leq Y$ and $\norm{F \sqrt{Y^{\dagger}} } \leq 1$, $X V_0 = \mathbf{0}$ by contraposition. First, we prove the forward implication. Suppose $X V_0 \neq \mathbf{0}$. We note that 
\[ XV_0 = \mathbf{0} \Leftrightarrow F^T F V_0 = \mathbf{0} \Leftrightarrow FV_0 = \mathbf{0}, \]
whence
\[ XV_0 \neq \mathbf{0} \Leftrightarrow FV_0 \neq \mathbf{0} \Leftrightarrow V_0 ^T F^T F V_0 = V_0 ^T X V_0 \Geq \mathbf{0}. \]
So $\exists y \neq 0$ such that $y^T V_0 ^T X V_0 y > 0$. We now define $z = V_0 y$ and observe that $Y z = 0$ (since $z$ is in the nullspace of $Y$, spanned by $V_0$), whence
\[ z^T (Y - X) z = - z^T X z = - y^T V_0 ^T X V_0 y  < 0, \]
implying that $X \not \Leq Y$.

\noindent Now suppose, instead, that $\norm{F \sqrt{Y^{\dagger}}} \not \leq 1$:
\[ \norm{F \sqrt{Y^{\dagger}} } \not \leq 1 \Leftrightarrow \sqrt{Y^{\dagger}} ^T F^T F \sqrt{Y^{\dagger}} \not \Leq I  \Rightarrow F^T F \not \Leq Y. \]
Since $F^T F = X$, it follows that $X \not \Leq Y$ and establishes the forward implication in lemma \ref{basic lemma}.

\noindent Now we prove the backward implication. Let $X \not \Leq Y$. Then $Y - X \not \Geq 0$ and $\exists z \neq 0$ such that
\[ z^T Y z < z^T X z. \]
Let us write, without loss of generality, $z = z_1 + z_2$, where $z_1 = V_+ y_1, z_2 = V_0 y_2$. Then 
\begin{align*}
z^T Y z = z_1 ^T Y z_1 < z^T X z &= z_1 ^T X z_1 + z_2 ^T X z_2 + 2 z_1 ^T X z_2 \quad (\text{since }X \text{ is symmetric})\\
&= z_1 ^T X z_1 + y_2 ^T V_0 ^T X V_0 y_2 + 2 z_1 ^T X V_0 y_2.
\end{align*}
We note that this cannot be true if both $XV_0 = \0$ (whence $X V_0 y_2 = 0$) and $z_1 ^T Y z_1 > z_1 ^T X z_1$. So either $XV_0 \neq \0$ or $z_1 ^T Y z_1 < z_1 ^T X z_1$. The former immediately negates $XV_0 = \mathbf{0}$. Suppose the latter is true. It follows that
\[ z_1 ^T \sqrt{Y^{\dagger}}^T Y \sqrt{Y^{\dagger}} z_1 = z_1 ^T z_1 = \norm{z_1}^2 < z_1 ^T \sqrt{Y^{\dagger}}^T X \sqrt{Y^{\dagger}} z_1. \]
We simply define $\hat{z_1} = z_1 / \norm{z_1}$ now to get 
\[ 1 < \hat{z_1} \sqrt{Y^{\dagger}}^T X \sqrt{Y^{\dagger}} \hat{z_1} = \hat{z_1} \sqrt{Y^{\dagger}}^T F^T F \sqrt{Y^{\dagger}} \hat{z_1}, \]
whence
\[ \norm{F \sqrt{Y^{\dagger}}} > 1,\]
negating $\norm{F \sqrt{Y^{\dagger}}} \leq 1$, proving the backward implication and establishing the equivalence promised in lemma \ref{basic lemma}.
\end{proof}

\subsection{Proof of Theorem \ref{main theorem}}
\begin{proof}
We recall from Section 2 that $\beta \geq \norm{C_{\N}}$ is the achievable error, and that $\norm{{C} - {Z}} \leq \beta$ only if $\norm{\tilde{C} - \tilde{Z}} \leq 1$, where $\tilde{C} = C_{\B} \sqrt{\Delta^{\dagger}}$, and $\tilde{Z} = Z_+{\Sigma}^{-1/2}_+V_+^T$. The least-rank solution to the latter inequality can be found by collecting singular value components of $\tilde{C}$ that are strictly larger than 1, i.e., $\tilde{Z} = \tilde{C}_{\tau}$, where $\tau$ is the number of singular values of $\tilde{C}$ that are larger than 1. Since we transform $\tilde Z$ to obtain $C_{\B, k}$, this allows us to establish a relationship between $\beta$ and the rank-restriction, $k$, to prove the three cases mentioned in theorem \ref{main theorem}. We establish this relationship in lemma \ref{lambda plus}, prove the cases mentioned in the theorem, and then complete the discussion by proving the lemma itself.\\  
\noindent First, let $ \sigma^{(\alpha)}(K)$ denote the  number of singular values of matrix $K$ that are larger than $\alpha$, and $\lambda_{+}(K)$, the number of positive eigenvalues of $K$.
So $\sigma^{(1)}(K) = \lambda_{+}(K^T K - I)$ and
\[ \sigma^{(\alpha)}(K) =  \lambda_{+}( K^T K - \alpha ^2 I) \quad \forall r > 0. \]

\noindent Now let $h  = rank(C_{\B} V_0 V_0 ^T)$ be as defined in lemma \ref{achievable}. Then, 
\begin{lemma}\label{lambda plus}
\[ \beta > \norm{C_{\N}} \Rightarrow \left\{ \begin{array}{l} \sigma^{(1)}(\tilde{C}) = \sigma^{(\beta)} (C) \\ h = 0 \end{array} \right. , \quad \beta = \norm{C_{\N}} \Rightarrow \left\{ \begin{array}{l} \sigma^{(1)}(\tilde{C}) \leq \sigma^{(\beta)} (C) \\ h \geq 1 \end{array} \right. \]
\end{lemma}

\subsubsection{Proving $\norm{C - C_{B, k}} = \sigma_{k+1}(C)$, for $k < k^*$}
\begin{proof}
For $k < k^*$, let $\beta = \sigma_{k+1}(C) \leq \sigma_{k^*} (C) < \norm{C_{\N}}$ be the error we look to achieve with an approximation $Z = B \G$. Clearly, $\sigma^{(\beta)} (C) = k$ and by lemma \ref{lambda plus}, we have
\[ \sigma^{(1)}(\tilde{C}) = k, \quad h = 0. \]
By lemma \ref{achievable}, there exists a $Z=(C_{\B} \sqrt{\Delta^{\dagger}})_k \sqrt{\Delta} +C_{\B} V_0V_0^T$ such that $\norm{C - Z} \leq \beta$. Since $h = rank(C_{\B} V_0V_0^T) = 0$, $rank(Z) = k$. However, by Eckart-Young theorem, $\forall D$ with $rank(D) \leq k$, $\norm{C - D} \geq \sigma_{k+1} (C)$, implying that
\[ \norm{C - Z} = \sigma_{k+1}(C).\]
So $C_{\B, k} = Z$ provides us the promised result.
\end{proof}

\subsubsection{Proving $\norm{C - C_{\B, k}} \leq (1 + \epsilon) \norm{C_{\N}}$ for $k^* \leq k \leq k^* + h$}
\begin{proof}
This follows by setting $\beta = (1 + \epsilon)\norm{C_{\N}}$ and arguing as in the immediately preceding case, without the use of the observation by Eckart and Young.
\end{proof}

\subsubsection{Proving $\norm{C- C_{\B, k}} = \norm{C_{\N}}$, for $k \geq k^* + h$}
\begin{proof}
Let $\beta = \norm{C_{\N}}$. We have, from lemma \ref{achievable} that $Z = (C_{\B} \sqrt{\Delta^{\dagger}})_k \sqrt{\Delta} +C_{\B} V_0V_0^T$ has the property that $\norm{C - Z} \leq \beta = \norm{C_{\N}}$, and 
\[ rank(Z) = \sigma^{(1)}(\tilde{C}) + h. \]
From lemma \ref{lambda plus}: $\sigma^{(1)}(\tilde C) \leq \sigma^{(\beta)}(C) \leq k^*$. So it follows that $\forall k \geq k^* + h,$ $rank(Z) \leq k$. Since $C_{\B}$ is the projection of $C$ onto $\B$, we have that for any $D$ with columns in $\B$,
\[ \norm{C - D}  \geq \norm{C - C_{\B}} = \norm{C_{\N}}.\]
Together with $\norm{C - Z}  \leq \norm{C_{\N}}$, setting $C_{\B, k} = Z$ proves the promised result.
\end{proof}

\subsubsection{Proof of Lemma \ref{lambda plus}}
\begin{proof}
We begin by observing that
\[ C^T C - \beta^2 I  = C_{\B} ^T C_{\B} + C_{\N} ^T C_{\N} - \beta^2 I = C_{\B} ^T C_{\B}  - \Delta, \]
and hence
\[ \sigma^{(\beta)} (C) = \lambda_+ \left( C^T C - \beta^2 I \right) = \lambda_+ \left( C_{\B} ^T C_{\B}  - \Delta \right). \]
We write the SVD of $\Delta$ as $[V_+, V_0] \Sigma [V_+, V_0]^T$, where the subscripts $+, 0$ stand for the basis of non-zero action (corresponding to positive singular values) and the basis for the null-space (corresponding to singular values that are 0). We now define 
\[ \Delta_1 = \Delta + V_0 V_0 ^T, \]
effectively replacing the zeros in $\Sigma$ by ones. Clearly, $\Delta_1 ^{-1} \Delta = \Delta^{\dagger} \Delta = V_+ V_+ ^T = I - V_0 V_0 ^T$. Since $\Delta_1$ is non-singular by construction, it follows from Sylvester's Law of Inertia that
\[ \sigma^{(\beta)} (C) = \lambda_+ (C_{\B} ^T C_{\B}  - \Delta) = \lambda_+ \left( \sqrt{\Delta_1 ^{-1}} ^T C_{\B} ^T C_{\B} \sqrt{\Delta_1 ^{-1}} -  I + V_0 V_0 ^T  \right) . \numberthis \label{beta station} \]
We now recall a standard result that $\forall G \Geq 0$, $\lambda(C + G) \geq \lambda(C)$. As a result, $\lambda_+ (C+G) \geq \lambda_+ (C)$, and we may proceed to reason as follows. By definition of $\Delta_1$, $\sqrt{\Delta_1^{-1}} = \sqrt{\Delta^{\dagger}} + V_0 V_0 ^T$, whence 
\[ \sqrt{ \Delta_1 ^{-1} } ^T C_{\B} ^T C_{\B} \sqrt{ \Delta_1 ^{-1}}  = \sqrt{\Delta^{\dagger}}^T C_{\B} ^T C_{\B} \sqrt{\Delta^{\dagger}} + H + G + G ^T, \text{ where } \]
\[ H = V_0 V_0 ^T C_{\B} ^T C_{\B} V_0 V_0 ^T \Geq 0, \quad  G = \sqrt{\Delta^{\dagger}}^T C_{\B} ^T C_{\B} V_0 V_0 ^T \Geq 0, \quad  G ^T \Geq 0 \] 
since these are simply products of SPSD matrices. Since $V_0 V_0 ^T \Geq 0$ as well, 
\begin{align*}
\lambda_+ \left( \sqrt{ \Delta_1 ^{-1} } ^T C_{\B} ^T C_{\B} \sqrt{ \Delta_1 ^{-1} } - I + V_0 V_0 ^T \right)  & \geq \lambda_+ \left( \sqrt{\Delta^{\dagger}}^T C_{\B} ^T C_{\B} \sqrt{\Delta^{\dagger}}  - I \right),  \numberthis \label{p2inv}\\ 
& \quad \text{ since } H + G + G ^T + V_0 V_0 ^T \Geq 0 \\
& \geq \lambda_+ \left( \tilde{C} ^T \tilde{C} - I \right) \numberthis \label{p2inv2} \\
& \quad \text{ since, by definition, } \tilde{C} = C_{\B} \sqrt{\Delta ^{\dagger}} \\
& = \sigma^{(1)}(\tilde C).
\end{align*}
Along with (\ref{beta station}), we now have
\[ \sigma^{(\beta)}(C) \geq \sigma^{(1)}(\tilde C). \numberthis \label{bto1} \]
Finally, we note that, when $\beta > \norm{C_{\N}}$, $\Delta = \beta ^2 I - C_{\N}^T C_{\N}  \> \mathbf{0}$ and invertible. So $V_0 V_0 ^T = \mathbf{0}$ whence $H, G, G^T = \mathbf{0}$, $h = rank(C_{\B} V_0 V_0 ^T) = 0$, and (\ref{p2inv}), (\ref{p2inv2}) and (\ref{bto1}) hold with equality. When $\beta = \norm{C_{\N}}$, $\Delta$ has a non-trivial null-space, implying that $V_0 V_0 ^T \neq \mathbf{0}$ and that $h \geq 1$. This completes the proof of lemma \ref{lambda plus}.
\end{proof}

\noindent To summarize: we've shown above that there is an approximation to $C$, $C_{\B, k^* + h}$, of rank at most $k^* + h$ with columns in the column space of $B$ such that $\norm{C - C_{\B, k^* + h}} = \norm{C_{\N}}$; we've also shown that, for all $k < k^*$, we may find an approximation, $C_{\B, k}$, with columns in the column space of $B$, such that $\norm{C - C_{\B, k}} = \sigma_{k+1}$. This closes the proof of Theorem \ref{main theorem}.
\end{proof}

\section{Discussion}
In this work, we have provided an efficient algorithm (Algorithm \ref{alg:outline}) to compute the best rank-$k$ approximation, $C_{\B, k}$, in spectral norm of a matrix, $C$, with columns belonging to the column-space, $\B$, of $B$. The running time of this algorithm is clearly dominated by the computation of the SVD of $C$. In the case where $rank(B) \ll rank(C)$, it would be desirable to obtain $C_{\B, k}$ in running time dominated by $T_{svd(B)}$. While we are currently unable to report such an algorithm, we propose the following intuitive conjecture which may serve as a direction of advancement towards obtaining an algorithm running in $O(T_{svd(B)} + mnr)$ as opposed to $O(T_{svd(C)} + mnr)$: We note that $(C_{\B})_k$ may be computed in $O(mnr + T_{svd(B)})$ (as in section 2.2 of \cite{BDM}). If $\B$ contains a good approximation to the top $k$ left singular vectors of $C$, it seems intuitive to expect $(C_{\B})_k$ to be a good approximation to $C_{\B, k}$. Precisely, we conjecture that\\
\textbf{Conjecture 1:} For some $p > 0$ and $\epsilon \in (0, 1)$:\\
\[ \frac{\norm{{C} - C_{\B, k}}}{\norm{C - C_k}} \leq (1+\epsilon)^p \quad \Rightarrow \quad
\frac{\norm{{C} - (C_{\B})_k} }{\norm{C - C_{\B, k}}}\leq (1+o(\epsilon)). \]
This provides an exact condition under which we may obtain an algorithm for computing a good approximation to $C_{\B, k}$ in $o(T_{svd(C)})$.\\

Finally: we make the assumption that we have an infinite precision machine for arithmetic operations, i.e., we can perform arithmetic operations in $O(1)$ time. In practice, when the minimum achievable error is $\norm{C_{\N}}$, we can get an error arbitrarily close to $\norm{C_{\N}}$. If we want to attain a specific relative error, $\epsilon$,  then the overall running time will be a factor $O\Big(\text{log}(\frac{1}{\epsilon})\Big)$ more.  
\bibliographystyle{abbrv}    % bibliography using BibTex format
\bibliography{reference}


\end{document}

\section{Appendix}
\subsection{Algorithm}
%% ============ The Algorithm begins ==================
\begin{framed}
\begin{alg}\label{alg:outline}
{\bf Opt\_Spectral\_Matrix\_Approx($A$, $B$, $k$, $\epsilon$) }
\end{alg}
\noindent {\bf Input:} $A \in \reals^{m \times n}$, $B \in \reals^{m \times r}$, a $k$ : $1 \leq k \leq n$, accuracy parameter $\epsilon > 0$.\\
\noindent {\bf Output:} $Z$, such that $Z=B X$, $\rank(Z) \leq k$, and error $\beta = \norm{A - Z}$ achieved according to Theorem \ref{main theorem}.
%\noindent {\bf Steps:}
%
\begin{align*}
1. \quad& \text{Compute $U_A, \Sigma_A, V_A$, such that, $ A = U_A\Sigma_A V_A^T$, $U_B, \Sigma_B, V_B$, such that $ B = U_B\Sigma_B V_B^T$,}\\ 
& \text{$A_B = BB^{\dagger}A$, and $A_N = A - A_B$}, \\
2. \quad& \text{Let, $h=\rank(A_BV_0V_0^T)$ }, \text{and set $\beta$ according to Theorem \ref{main theorem}};\\
3. \quad& \text{Let }\Delta = (\beta^2 I_n - A_N^TA_N); \text{Compute $U_{\Delta}, \Sigma_{\Delta}, V_{\Delta}$, such that, $ \Delta = U_{\Delta}\Sigma_{\Delta} V_{\Delta}^T$ and}, \\
& V_{\Delta}=[V_+ \quad \tilde{V}_0], \text{where $\tilde{V}_0 = V_0$ if $\beta = \norm{A_N}$, otherwise $\tilde{V}_0 = \emptyset$}; \\
4. \quad & \text{Let $\Sigma_{\Delta} = [\Sigma_+ \quad \Sigma_0]$ where $\Sigma_+$ corresponding to $V_+$}.\\
5. \quad& \text{Return } Z = (A_BV_+{\Sigma}^{-1/2}_+V_+^T)_{k-h}V_+{\Sigma}^{1/2}_+V_+^T + A_BV_0V_0^T \text{ and } \beta;
\end{align*}
%
\end{framed}
%% ============ The Algorithm ends ==================

\noindent \textbf{Computational complexity}: \quad Computing $A_B$ takes $O(T_{svd(B)}+mnr)$. We need SVD of $A$ to locate $\norm{A_N}$ in the spectrum of $A$. This takes $O(T_{svd(A)}+ log (\rho))$. $\sqrt{\Delta}$ and ${{\Delta}^{\dagger}}^{\frac{1}{2}}$ can be computed using SVD of $\Delta$. We need SVD of $A_BV_+{\Sigma}^{-1/2}_+V_+^T$ to compute $Z$. Multiplication of two $n \times n$ matrices takes time $O(n^\omega), \omega < 3$. Thus, $T_{svd(A)}$ dominates the runtime of the algorithm, making it a $O(T_{svd(A)})$ algorithm. \\

\section{Ending discussion for proof of main theorem}
To see what happens in the region in between, where the rank, $k$, of the approximant is such that $k^* \leq k < k^* + h$, we reason as follows. For all $\epsilon > 0$, define $\beta(\epsilon) = (1 + \epsilon) \norm{C_N}$. Clearly, $\Delta(\epsilon) = \beta(\epsilon) ^2 I - C_N ^T C_N \succ  0$, invertible, and we may proceed the same way as shown previously to assert that there is an approximant, $C_{B, k^*}$, with the property that $\norm{C - C_{B, k^*}} = \beta(\epsilon), \forall \epsilon > 0$. Since $k^*$ is the smallest rank for which this property holds (for rank $ k < k^*$, we cannot get a lesser error than $\sigma_{k+1} > \norm{C_N}$), and we require the rank of the approximant to be at least $k^* + h$ for when $\epsilon = 0$, it follows that $\forall k^* \leq k < k^* + h$, we can do no better than incur the error $(1 + \epsilon) \norm{C_N}$ for $\epsilon > 0$. 

\section{Right After Lemma 5}
\begin{lemma}\label{kstar h}
We can find $Z \in B$, with $rank(Z) = k^* + h$, such that, $\norm{C - Z} = \norm{C_N}$.\\
Furthermore, 
\[ Z = \Big((BB^{\dagger}C)V_+{\Sigma}^{-\frac{1}{2}}_+V_+^T\Big)_{k^*}V_+{\Sigma}^{\frac{1}{2}}_+V_+^T + (BB^{\dagger}C)V_0V_0^T\]
\end{lemma}
\noindent We can use similar arguments to have the following lemmas.
\begin{lemma}\label{k less kstar}
If \text{ }$k < k^*$,
\[ \norm{C - \Pi_{B, k}(C)} = \sigma_{k+1}(C), \] and
\[ \Pi_{B, k}(C) = \Big((BB^{\dagger}C)V_+{\Sigma}^{-\frac{1}{2}}_+V_+^T\Big)_{k}V_+{\Sigma}^{\frac{1}{2}}_+V_+^T\]
\end{lemma}
\noindent \\
However, when $k^* \leq k < k^*+h$, we may not achieve the minimum error $\norm{C_N}$. Instead, we can obtain an approximation with error arbitrarily close to $\norm{C_N}$.
\begin{lemma}\label{k less kstar h}
Let $\text{ }\beta = (1 + \epsilon)\norm{C_N}, \quad \epsilon > 0$. Then, 
\[ \Pi_{B, k}(C) = \Big(BB^{\dagger} C {\Delta}^{-\frac{1}{2}}_{\epsilon}\Big)_k {\Delta}^{\frac{1}{2}}_{\epsilon}\]
 where \quad $\Delta_{\epsilon}  = (1+\epsilon)^2\norm{C_N}^2I_n - C_N^TC_N$.
\end{lemma}
\noindent \\